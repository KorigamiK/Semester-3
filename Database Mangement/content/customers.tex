\section{Customers}

\subsection{Schema}

Consider the following relational schema:

\begin{sqlSchema}{customers}

    CUSTOMERS (|\key{cust\_num}|, |\text{cust\_lname}|, |\text{cust\_fname}|, |\text{cust\_balance}|)

    PRODUCT (|\key{prod\_num}|, |\text{prod\_name}|, price)

    INVOICE (|\key{inv\_num, prod\_num, cust\_num, inv\_date}|, |\text{unit\_sold}|, |\text{inv\_amount}|)

\end{sqlSchema}

\subsection{Table Definitions}

\begin{enumerate}[label=\alph*)]
    \item Create the tables for the schema

          \inputsql{customers}

          \begin{lstlisting}[style=output]
practicals=# \dt
    List of relations
Schema |   Name   | Type  |  Owner
--------+----------+-------+----------
public | customers | table | postgres
public | invoice   | table | postgres
public | product   | table | postgres
(3 rows)
          \end{lstlisting}

    \item Insert the following tuples into the tables

          \inputsql{customers_data}

\end{enumerate}

\subsection{Queries}

Write SQL queries and relational algebraic expression for the following:

\begin{enumerate}
    \item Find the names of the customer who have purchased no item. Set default value of cust\_balance as 0 for
          such customers.

          \begin{enumerate}
              \item Relational Algebra

                    \begin{equation*}
                        \pi_{\text{cust\_lname} + " " + \text{cust\_fname}}(\sigma_{\text{cust\_balance} = 0} (CUSTOMERS))
                    \end{equation*}
                    \pagebreak

              \item SQL

                    \begin{sqlQuery}{customers1}
                        SELECT concat(cust_lname , ' ' , cust_fname) as name
                        FROM CUSTOMERS
                        WHERE cust_balance = 0;
                    \end{sqlQuery}

              \item Output

                    \begin{lstlisting}[style=output]
practicals=# \i /home/sql/customers1.sql
    name    
------------
 Smith John
 Smith Mary
(2 rows)
                    \end{lstlisting}

          \end{enumerate}

    \item Write the trigger to update the \textsc{CUST\_BALANCE} in the CUSTOMER table when a new invoice record
          is entered for the customer.


          \begin{sqlQuery}{customers2}
                CREATE TRIGGER update_cust_balance
                AFTER INSERT ON INVOICE
                FOR EACH ROW
                BEGIN
                    UPDATE CUSTOMERS
                    SET cust_balance = cust_balance + NEW.inv_amount
                    WHERE cust_num = NEW.cust_num;
                END;
        \end{sqlQuery}


    \item Find the customers who have purchased more than three units of a product on a day.

          \begin{enumerate}
              \item Relational Algebra

                    \begin{multline*}
                        \pi_{\text{cust\_lname} + " " + \text{cust\_fname}} (\sigma_{\text{unit\_sold} \geq  3} \\
                        (CUSTOMER \bowtie \sigma_{\text{unit\_sold}}\\
                        ({}_{\text{cust\_num, inv\_date} > \text{prod\_num}} \; \mathcal{F}_{\; sum(\text{unit\_sold})} (INVOICE))))
                    \end{multline*}

              \item SQL

                    \begin{sqlQuery}{customers3}
                        SELECT concat(cust_lname , ' ' , cust_fname) as name
                        FROM CUSTOMERS
                        WHERE cust_num IN
                        (
                            SELECT cust_num
                            FROM INVOICE
                            GROUP BY cust_num, inv_date, prod_num
                            HAVING sum(unit_sold) >= 3
                        );
                    \end{sqlQuery}
                    \pagebreak

              \item Output

                    \begin{lstlisting}[style=output]
practicals=# \i /home/sql/customers3.sql
    name    
------------
 Smith John
 Jones Mary
(2 rows)
                    \end{lstlisting}

          \end{enumerate}

    \item Write a query to illustrate Left Outer, Right Outer and Full Outer Join.

          \begin{enumerate}
              \item Left Outer Join
                    \begin{equation*}
                        CUSTOMER ]\bowtie INVOICE
                    \end{equation*}

                    \begin{sqlQuery}{customer4a}
											SELECT CONCAT(C.cust_fname, c.cust_lname) as name,
											LEFT JOIN INVOICE i
											ON C.cust_num=i.cust_num
										\end{sqlQuery}

              \item Right Outer Join

                    \begin{equation*}
                        CUSTOMER \bowtie[ INVOICE
                    \end{equation*}

                    \begin{sqlQuery}{customer4b}
											SELECT CONCAT(C.cust_fname, c.cust_lname) as name,
											RIGHT JOIN INVOICE i 
											ON C.cust_num=i.cust_num
										\end{sqlQuery}

              \item Full Outer Join

                    \begin{equation*}
                        CUSTOMER ]\bowtie[ INVOICE
                    \end{equation*}

                    \begin{sqlQuery}{customer4c}
                    SELECT CONCAT(C.cust_fname, " ", C.cust_lname) as name
                    LEFT JOIN INVOICE i
                    ON C.cust_num=i.cust_num
                    UNION
                    SELECT CONCAT(C.cust_fname, " ", C.cust_lname) as name, i.inv_amount
                    RIGHT JOIN INVOICE i
                    ON C.cust_num=i.cust_num
                    \end{sqlQuery}

          \end{enumerate}
          \pagebreak

    \item Count number of products sold on each date.

          \begin{enumerate}
              \item Relational Algebra

                    \begin{equation*}
                        \pi_{\text{inv\_date}, sum(\text{unit\_sold})}({}_{\; \text{inv\_date}} \; \mathcal{F}_{\; sum(\text{unit\_sold})} (INVOICE))
                    \end{equation*}

              \item SQL

                    \begin{sqlQuery}{customers5}
                        SELECT inv_date, sum(unit_sold)
                        FROM INVOICE
                        GROUP BY inv_date
                    \end{sqlQuery}

              \item Output

                    \begin{lstlisting}[style=output]
practicals=# SELECT inv_date, sum(unit_sold)
    FROM INVOICE
    GROUP BY inv_date;
  inv_date  | sum 
------------+-----
 2015-10-01 |  10
 2015-07-01 |   7
 2015-03-01 |   3
 2015-02-01 |   2
 2015-01-01 |   8
 2015-06-01 |  14
 2015-05-01 |   5
 2015-11-01 |   1
 2015-04-01 |  13
(9 rows)
                \end{lstlisting}


          \end{enumerate}

    \item As soon as customer balance becomes greater than Rs. 100,000, copy the customer\_num in new table
          called "GOLD\_CUSTOMER"

          \begin{enumerate}
              \item Create table GOLD\_CUSTOMER

                    \begin{sqlQuery}{customers6a}
                        CREATE TABLE GOLD_CUSTOMER
                        (
                            cust_num int,
                            PRIMARY KEY (cust_num),
                            FOREIGN KEY (cust_num) REFERENCES CUSTOMERS (cust_num)
                        )
                    \end{sqlQuery}
                    \pagebreak

              \item Create a trigger to update the GOLD\_CUSTOMER table when a new invoice record is entered for the customer.

                    \begin{sqlQuery}{customers6b}
                        CREATE TRIGGER update_gold_customer
                        AFTER INSERT ON INVOICE
                        FOR EACH ROW
                        BEGIN
                            IF NEW.cust_balance > 100000 
                            AND NEW.cust_num NOT IN (SELECT cust_num FROM GOLD_CUSTOMER) THEN
                                INSERT INTO GOLD_CUSTOMER VALUES (NEW.cust_num);
                            END IF;
                        END                        
                    \end{sqlQuery}

          \end{enumerate}

    \item Add a new attribute CUST\_DOB in customer table

          \begin{sqlQuery}{customers7}
            ALTER TABLE CUSTOMERS
            ADD COLUMN cust_dob date
          \end{sqlQuery}

\end{enumerate}
