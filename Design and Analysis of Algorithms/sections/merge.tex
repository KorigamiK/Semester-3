\section{Merge Sort}

\subsection{Description}
Merge sort is similar to the quick sort algorithm as it uses the divide and conquer
approach to sort the elements. It is one of the most popular and efficient sorting
algorithm. It divides the given list into two equal halves, calls itself for the two halves
and then merges the two sorted halves. We have to define the merge() function to
perform the merging.

The sub-lists are divided again and again into halves until the list cannot be divided
further. Then we combine the pair of one element lists into two-element lists, sorting
them in the process. The sorted two-element pairs is merged into the four-element lists,
and so on until we get the sorted list.


\subsection{Algorithm}

% 1.	mergeSort(array, left, right)
% 2.	if left < right
% 3.	middle = (left + right) / 2
% 4.	mergeSort(array, left, middle)
% 5.	mergeSort(array, middle + 1, right)
% 6.	merge(array, left, middle, right)

\begin{algorithm}[H]
    \caption{Merge Sort}
    \begin{algorithmic}[1]
        \Procedure{MergeSort}{$array, left, right$}
        \If{$left < right$}
        \State $middle$ = $(left + right) / 2$
        \State MergeSort($array, left, middle$)
        \State MergeSort($array, middle + 1, right$)
        \State merge($array, left, middle, right$)
        \EndIf
        \EndProcedure
    \end{algorithmic}
\end{algorithm}

\subsection{Code}

\inputminted[fontsize=\footnotesize,bgcolor=bg,linenos,autogobble,frame=single,framerule=0.01pt,rulecolor=FSBorder,stripall,breaklines]{c++}{code/merge.cpp}
\subsection{Output}

\begin{lstlisting}[style=output]
Enter the number of elements: 5
Enter the elements: 170 45 75 90 802
Sorted array is 45 75 90 170 802
\end{lstlisting}
