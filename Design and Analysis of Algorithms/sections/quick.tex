\section{Quick Sort}

\subsection{Description}
Quicksort picks an element as pivot, and then it partitions the given array around the
picked pivot element. In quick sort, a large array is divided into two arrays in which one
holds values that are smaller than the specified value (Pivot), and another array holds
the values that are greater than the pivot.
After that, left and right sub-arrays are also partitioned using the same approach. It will
continue until the single element remains in the sub-array.

\subsection{Algorithm}

% 1.	quickSort(array, left, right)
% 2.	if left < right
% 3.	pivot = partition(array, left, right)
% 4.	quickSort(array, left, pivot - 1)
% 5.	quickSort(array, pivot + 1, right)
%

\begin{algorithm}[H]
    \caption{Quick Sort}
    \begin{algorithmic}[1]
        \Procedure{QuickSort}{$array, left, right$}
        \If{$left < right$}
        \State $pivot$ = partition($array, left, right$)
        \State QuickSort($array, left, pivot - 1$)
        \State QuickSort($array, pivot + 1, right$)
        \EndIf
        \EndProcedure
    \end{algorithmic}
\end{algorithm}

\subsection{Code}

\inputminted[fontsize=\footnotesize,bgcolor=bg,linenos,autogobble,frame=single,framerule=0.01pt,rulecolor=FSBorder,stripall]{c++}{code/quick.cpp}

\subsection{Output}

\begin{lstlisting}[style=output]
Enter the number of elements: 5
Enter the elements: 170 45 75 90 802
Sorted array is 45 75 90 170 802
\end{lstlisting}